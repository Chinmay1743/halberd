\documentclass[a4paper]{book}
\usepackage[english]{babel}
\begin{document}

\title{Halberd user's guide}
\author{Juan M. Bello Rivas}
\maketitle

\tableofcontents

% 1. Explain the problem being solved.
% 2. Present the concepts, not just the features.
% 3. Give'em more than they deserve.
% 4. Make it enjoyable to read.

\chapter{Introduction}

Halberd discovers HTTP load balancers. It is useful for testing load balancer
configurations and for web application auditing purposes. 

\section{Motivation}

To cope with heavy traffic loads, web site administrators often install load
balancer devices. These machines hide (possibly) many real web servers behind
a virtual IP. They receive HTTP requests and redirect them to the real web
servers in order to share the traffic between them.

There are a few ways to map the servers behind the VIP and to reach them
individually.

Identifying and being able to reach all real servers individually (effectively
bypassing the load balancer) is very important for an attacker trying to break
into a site. It is often the case that there are configuration differences
ranging from the slight:

\begin{itemize}
  \item server software versions,
  \item server modules
\end{itemize}

to the extreme:

\begin{itemize}
  \item different platforms
  \item server software.
\end{itemize}

For an attacker, this information is crucial because he might find vulnerable
configurations that otherwise (without mapping the real servers) could have
gone unnoticed.

But someone trying to break into a web site doesn't have server software as its
only target. He will try to subvert dynamic server pages in several ways.  By
identifying all the real servers and scanning them individually for
vulnerabilities, he might find bugs affecting only one or a few of the web
servers. Even if all machines are running the same server software, halberd can
enumerate them allowing more thorough vulnerability scans on the application
level.

\chapter{Concepts}

Halberd operates in stages:

\begin{enumerate}
  \item Initially, it sends multiple requests to the target web server and
  records its responses.  This is called the \emph{sampling phase}.  The time
  to spend in this phase and the amount of HTTP requests to be sent can be
  specified using the \texttt{--time} and \texttt{--parallelism} command line
  options.  See \ref{sec:time} and \ref{sec:parallelism}.

  \item After the analysis phase finishes, either normally or because the user
  interrupted it pressing Control-C, the program processes the replies looking
  for signs of load balancing.  This is called the \emph{analysis phase}.

  \item Finally, halberd writes a report of its findings to the screen (or to a
  file if the \texttt{--out} option is used).
\end{enumerate}

The user may skip the scanning phase and proceed to the analysis of samplings
saved in a previous session. See \ref{sec:read} and \ref{sec:write}.

The following is a list of detection techniques currently implemented:

\section{Date comparison}

HTTP responses reveal the internal clock of the web server that produces them.
By tracking how many different clocks appear to be, halberd can have some
insight into the number of real servers.

\section{MIME header field names, values and their order}

Differences in fields appearing in server responses can allow halberd to
narrow down its search.

%\section{Special cookie usage}
%
%Some HTTP load balancers generate a cookie which will be used to direct future
%requests to a certain server. Administrators can also set up cookies that will
%be taken into account for deciding where to send requests. By selecting which
%cookies to send, the program attempts to force the load balancer to route
%requests to a server of our choice.

\section{Generating high amounts of traffic}

Under certain configurations, load balancers might start to distribute traffic
only after a certain threshold has been reached.  By default, halberd attempts
to generate a sizeable traffic volume to trigger this condition and reach as
many real servers as possible. See \ref{sec:parallelism}.

%Halberd can be used in a distributed fashion to generate heavy traffic loads.
%Running the program in RPC server mode in computers located at several
%networks lets the user launch distributed scans against a web site, forcing
%the idle web servers to reveal themselves.

\section{Using different URLs}

An HTTP load balancer can be configured to take URLs into account when deciding
where to direct requests. Gathering URLs with a spider and feeding them to
halberd will make it more likely that all servers will reply at some point.
See \ref{sec:urlfile}.

\section{Detecting server-side caches}

Halberd might come across web sites having server-side caches (e.g.: Squid).
This kind of configuration is appropriately identified by halberd and in case
there is more than one cache, the program can enumerate them.

\section{Obtaining public IP addresses}

Sometimes cookies or special MIME fields in server responses can reveal public
IP addresses or host names. In these cases the load balancer can be bypassed
connecting directly to the real servers.

\chapter{Installation}

\section{Prerequisites}

You need Python version 2.4 or above with the threading and MD5 (or SHA1)
modules enabled.  If you want to scan through HTTPS you will also need a Python
interpreter configured with support for SSL sockets.

\section{Supported platforms}

Halberd should work in every machine satisfying the prerequisites mentioned
above.  The program has been successfully built and tested on Linux, Windows
2000 and Mac OS X.

\section{Installation steps}

Installing halberd is a very simple task.  It suffices to write (perhaps as
root):

\begin{verbatim}
python setup.py install
\end{verbatim}

Halberd creates the file \verb|$HOME/.halberd/halberd.cfg| when you execute it
for the first time.  This file lets the user configure proxy settings and
specify an SSL certificate.

It is recommended that you read the output of
\verb|python setup.py install --help| in case you want to fine tune the
installation process.

\chapter{Operation}

This chapter explains how to use halberd in the real world.

\section{Sample session}

Peter is on the phone, talking to a client who asks him to perform a
penetration test of a large web application that's just entered production.

He impatiently taps his fingers against the wooden table waiting for the email
from the client to arrive.  Our hero is finally given the green light and he
begins his dutiful work.

First of all, a load balancer scanning is due.  He fires up a virtual terminal
and writes:

\begin{verbatim}
$ halberd http://www.target-company.com
\end{verbatim}

Peter reads the output of his previous incantation and nods...

% XXX: use sed or whatever to update the version here.

\begin{verbatim}
halberd 0.2.0

INFO looking up host www.target-company.com...
INFO host lookup done.
INFO www.target-company.com resolves to x.x.x.1
INFO www.target-company.com resolves to x.x.x.2
x.x.x.1   [######    ]  clues:   3 | replies:  17 | missed:   0
\end{verbatim}

After 17 replies he thinks halberd has enough samples to analyze, so he hits
Control-C and stops the scan for this host.  Peter notes there is DNS load
balancing in place so he'll have to check both IP addresses.

\begin{verbatim}

*** finished (received SIGINT) ***

=============================================================
http://www.target-company.com (x.x.x.1): 1 real server(s)
=============================================================

server 1: foo/1.2.3  mod_bar/4.2 (Unix)
-------------------------------------------------------------

difference: 3600 seconds
successful requests: 17 hits (100.00%)
header fingerprint: c0ba8262100168851872c8feea3196f21ba2d732
different headers:
  1. Server: foo/1.2.3  mod_bar/4.2 (Unix)
\end{verbatim}

"Nothing to see here, let's move along" muttered Peter while halberd
progressed to the second IP address.

\begin{verbatim}
============================================================
http://www.target-company.com (x.x.x.2): 2 real server(s)
============================================================

server 1: foo/1.2.2  mod_bar/4.2 (Unix)
------------------------------------------------------------

difference: 3600 seconds
successful requests: 11 hits (33.33%)
header fingerprint: 732deadbeef100168851872c8feea196f21ba2d2
different headers:
  1. Date: Wed, 16 Aug 2006 22:47:04 GMT
  2. Server: foo/1.2.2  mod_bar/4.2 (Unix)

server 2: foo/1.2.3  mod_bar/4.2 (Unix)
------------------------------------------------------------

difference: 3662 seconds
successful requests: 23 hits (66.66%)
header fingerprint: ad2d33a88f259b434c094a7b1172f5697a35cff4
different headers:
  1. Date: Wed, 16 Aug 2006 22:46:02 GMT
  2. Server: foo/1.2.3  mod_bar/4.2 (Unix)
\end{verbatim}

"Aha!" shouts our fellow, almost falling off his chair.  "Not only did they
forget to set up NTP on those servers, letting me distinguish them easily,
they also have different \emph{server versions}!"

His eyes go blank while he runs a search in his brain cell based, caffeine
fueled vulnerability database for the terms foo/1.2.2 mod\_bar/4.2.  He
remembers there was a recent exploit for a buffer overflow in that version of
foo.

Peter has to take into account that the vulnerable web server gets one third
of the traffic (33.33\%).  Thus, he executes the exploit enough times to make
sure it hits the exposed target and he finally breaks into the machine.

"I'm so lucky to have this wonderful load balancer detector in my toolkit!
This vulnerability could have easily been skipped."

Our hero begins to walk in circles around the room, planning the next stages
of his assault and anticipating with apprehension the time when he'll have to
write the report for his client.

\section{Command line options}

\subsection{\texttt{--version}}

Shows the program's version number.

\subsection{\texttt{-h, --help}}

Shows a help message describing every option.

\subsection{\texttt{-q, --quiet}}

Runs quietly, limiting the amount of information being displayed while the
program runs.

\subsection{\texttt{-d, --debug}}

Enables debugging information.  This can be useful if you want halberd to dump
all the HTTP headers or if you're debugging the tool itself.

\subsection{\texttt{-t NUM, --time=NUM}}

\label{sec:time}
Stops halberd after NUM seconds have passed since the beginning of the
sampling phase.

\subsection{\texttt{-p NUM, --parallelism=NUM}}

\label{sec:parallelism}
Specifies the number of parallel threads to use for network operations.  This
can increase the amount of requests per second during the sampling phase.

\subsection{\texttt{-u FILE, --urlfile=FILE}}

\label{sec:urlfile}
Read URLs from FILE.  FILE is a text archive containing an URL in each line.
halberd will scan these URLs one by one.

\subsection{\texttt{-o FILE, --out=FILE}}

Writes the human-readable results from the analysis phase to FILE.

\subsection{\texttt{-a ADDR, --address=ADDR}}

Specifies the target by its IP address.

\subsection{\texttt{-r FILE, --read=FILE}}

\label{sec:read}
Loads and analyzes \emph{clues} from FILE.  \emph{Clues} are what halberd uses
to figure out (during the analysis phase) whether there is a load balancer or
not.  These \emph{clues} can be written (un-analyzed) to a file with the
\texttt{--write} option for comparison with future scans or other purposes.

\subsection{\texttt{-w DIR, --write=DIR}}

\label{sec:write}
Saves \emph{clues} to the specified directory.  If it doesn't exist, it will be
created.

For being portable, \emph{clues} are stored in text files contained in a
special directory layout.  The following is an example of the generated file
hierarchy:

\begin{verbatim}
http___www_target_company_com/
http___www_target_company_com/x_x_x_1.clu
http___www_target_company_com/x_x_x_2.clu
\end{verbatim}

\subsection{\texttt{--config=FILE}}

Tells halberd to use the configuration stored in FILE instead of the default
\verb|halberd.cfg|.

\chapter{Support}

Suggestions, bug reports and patches can be emailed to the author at
\texttt{jmbr+halberd@superadditive.com}

\end{document}

% vim: ts=2 sw=2 et ft=tex
